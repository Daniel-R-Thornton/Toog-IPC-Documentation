\documentclass[english,10pt,a4paper]{report}
\usepackage[T1]{fontenc}
\usepackage{amssymb}
\usepackage{graphicx}
\usepackage{amsmath}
\usepackage{fancyhdr}
\usepackage{babel}
\title{MTH132 Assignment 1}
\author{Daniel Thornton 11748014 }

\begin{document}
	\pagestyle{fancy}
	\fancyhead{}
	\fancyhead[RO]{MTH132 Assignment 1}
\fancyfoot{} % clear all footer fields
\fancyfoot[LE,RO]{\thepage}
\fancyfoot[CO,RE]{Daniel Thornton 11748014 }
% Some content:
\maketitle

\section{Question 1}
\subsection*{a)}
	\hspace{2cm}\begin{minipage}{\dimexpr\textwidth-2cm}
	\indent	\subsubsection*{i)}
	$ \{x \vert x \in \mathbb{Z}	  \textmd{ and } x \textmd{ divisible by }  5\} $
	
	\subsubsection*{ii)}
	$ \{ x^2 \mid x \in C \} $
	
	\end{minipage}
	

\subsection*{b)}
\hspace{2cm}\begin{minipage}{\dimexpr\textwidth-2cm}
	\subsubsection*{i)}
	Integers greater than 10. 
	\\
	Example:
	$\{11,12,13,\textmd{…}\}$
	This set is countably infinite as there is no finite limit on numbers greater than 10.
	It is countable as we can put it into a one-to-one correspondence between it and the positive integer set. It can be mapped as follows:
	$F(n)= n + 10$
	Such so that
	$ \vert1\vert \textmd{ of } \mathbb{Z} \textmd{ is } f(1)=11 $
	
	\subsubsection*{ii}
	Example:
	\[\{0,0.001,0.00001,\textmd{…,}2\}\]
	
	This set is infinite as no matter how small the interval between numbers there will always be a smaller measure available. Due to the infinite nature of the set and the infinite number of intervals the set is therefore uncountable. This means that the set cannot successfully be mapped to integers.
\end{minipage}


\pagebreak
\subsection*{c)}
\begin{figure}[tbph!]
	\centering
	\includegraphics[width=0.7\linewidth]{screenshot003}
	\caption{In the following diagrams we can see the sub formulas, and how they interact to make up the both sides of the equation, from this we can see how A both sides of the equation match}
	\label{fig:screenshot003}
\end{figure}

\pagebreak

\subsection*{d)}


\hspace{2cm}
\begin{minipage}{\dimexpr\textwidth-2cm}
	\begin{enumerate}
		\item \textbf{Union:} \\
		\[
		A \cup B
		\]
		\[
		\{a,b,c,d,e,f,g,h\}
		\]
		\item \textbf{Intersection:} \\
		\[
		A \cap B
		\]
		\[
		\{a,b\}
		\]
		\item \textbf{Difference (A - B):} \\
		\[
		A - B
		\]
		\[
		\{c\}
		\]
		\item \textbf{Difference (B - A):} \\
		\[
		B - A
		\]
		\[
		\{e,f,g,h\}
		\]
	\end{enumerate}
\end{minipage}
	


\pagebreak
\section*{Question 2}
\subsection*{a)}
\begin{tabular}{|c|c|c|c|c|c|c|c|}
    \hline
	p & q & s $\neg$ p & $\neg$ q & p$\wedge$q & $\neg$ (p$\wedge$q) &p$\vee$ $\neg$q & $\neg$ (p$\wedge$q) $\wedge$(p $\vee$ $\neg$q)\\
	\hline
	0&0&1&1&0&1&1&1\\
	\hline
	1&0&0&1&0&1&1&1\\
	\hline
	1&1&0&0&1&0&1&0\\
	\hline
	0&1&1&0&0&1&0&0\\
    \hline
    
\end{tabular}\\ 
	
we can see from the truth table above that the values for $\neg$q and \\
$\neg$ (p$\wedge$q) $\wedge$(p $\vee$ $\neg$q) match under all conditions.
using the table above we can determine that p has no direct affect on the outcome.


\subsection*{b)}
\paragraph{Prove that :}$A \cup (B-C) = (A \cup B) - (\overline{A} \cap C)$\\
Note: $ A-B = A \cap \overline{B}\\$
\begin{flushright}
 $\textmd{replace (B-C) with }(B\cap\overline{C})$
\end{flushright}
\begin{flushleft}
$A \cup (B\cap\overline{C}) = (A \cup B) - (\overline{A} \cap C) $
\end{flushleft}
\begin{flushright}
Distributive rule
\end{flushright}
\begin{flushleft}  
$(A \cup B) \cap (A\cup \overline{C})= (A \cup B) - (\overline{A} \cap C)$
\end{flushleft} 
\begin{flushright}
	apply definition that A-B=A$\cap$ $\overline{B}$ in this case we  treat the first pair of brackets as A and the second as B ( A=>$(A \cup B)$ , B=> $(\overline{A} \cap C)$)
	
\end{flushright}

$(A \cup B) \cap (A\cup \overline{C})= {(A \cup B) \cap\overline{ (\overline{A} \cap C)}} $
\begin{flushright}
De Morgan's Laws $\overline{(\overline{A}\cap C)} =  \overline{\overline{A}} \cup \overline{C}$ \\
\end{flushright}

$(A \cup B) \cap (A\cup \overline{C})= (A \cup B) \cap (\overline{\overline{A}} \cup \overline{C}) $
\begin{flushright}
	Double compliment is just the original set, $\overline{\overline{A}} = A$\\
\end{flushright}
$(A \cup B) \cap (A\cup \overline{C})= (A \cup B) \cap (A \cup \overline{C}) $

\paragraph{From the logic applied above we can see that:\\ $A \cup (B-C) = (A \cup B) - (\overline{A} \cap C)$}

\subsubsection{C)}

$\neg q \wedge (p =>q)$

\begin{flushright}
	substitute p=>q with $\neg$ p $\vee$ q since p=>q is only true when p is false or q is true
\end{flushright}

$\neg q \wedge (\neg p\vee q)$


\begin{flushright}
	expand out the brackets using the distributive proposition
\end{flushright}
$ \neg q \wedge ( \neg p \vee q) = (\neg q\wedge \neg p ) \vee (\neg q \wedge q)$

\begin{flushright}
 	 $(\neg q \wedge q)$  is always false since both q and $\neg$q cannot both be true
\end{flushright}
$ (\neg q\wedge \neg p ) \vee F$

\begin{flushright}
	simplify the statement
\end{flushright}
$\neg q \wedge \neg  p$

\paragraph{
checking if is tautological}
\[
\begin{array}{|c|c|c|c|c|}
	\hline
	p & q & \neg p & \neg q & \neg q \wedge \neg p \\
	\hline
	1 & 1 & 0 & 0 & 0 \\
	1 & 0 & 0 & 1 & 0 \\
	0 & 1 & 1 & 0 & 0 \\
	0 & 0 & 1 & 1 & 1 \\
	\hline
\end{array}`
\]
	
\paragraph{As it is not true under all conditions this means it is not tautological, as in different states we get different results we can see that is is a contingency as different inputs produce differing results.}



\section*{Question 3}

\subsection*{a)}
\[
3.4 + 5.7 + 7.10 + ... + (2n+1)(3n+1) = \frac{n(4n^2 + 11n + 9)}{2} \quad \text{for} \quad n \geq 1.
\]\\

\subsubsection{Base Case}
prove for the smallest base case in this case we use n=1;

LHS = $(2 \cdot 1 + 1)(3 \cdot 1 +1) = 12$\\

do the same for RHS\\
RHS = $\frac{1(4\cdot 1 ^{2} + 11 \cdot 1 + 9 )}{2} = 12$\\

so: 

$LHS = RHS = 12,  n=1$
\subsubsection{Induction Hypothesis}

Assume that p(k) is true, i.e., we have

$3.4 + 5.7 + 7.10 + ... + (2k+1)(3k+1) = \frac{k(4k^2 + 11k + 9)}{2}$


\subsubsection{Induction Step}

Now consider\\
 $n = k + 1$ \\
  
 $LHS= \\= (3.4 + 5.7 + 7.10 + ... + (2k+1)(3k+1))+(2(k+1)+1)(3(k+1)+1)$ \\\\
 $= (3.4 + 5.7 + 7.10 + ... + (2k+1)(3k+1))+(2k+3)(3k+4)$\\\\
 $= \frac{k(4k^2 + 11k + 9)}{2} + (2k+3)(3k+4)$\\\\
 $=\frac{4k^{3} + 11k^{2} + 9k}{2} + (2k+3)(3k+4)$\\\\
$=\frac{4k^{3} + 11k^{2} + 9k + 2((2k+3)(3k+4))}{2}$\\\\
$=\frac{4k^{3} + 11k^{2} + 9k +12 k^{2} + 34 k + 24}{2}$\\\\
$=\frac{4 k^3+ 23 k^2+43 k + 24 }{2}$\\\\




$RHS =  $\\\\
$=\frac{(k+1)(4(k+1)^2 + 11(k+1) + 9)}{2}$\\\\
$=  \frac{(4(k+1)^3+ 11(k+1)^{2} + 9(k+1))}{2}$\\\\
$=  \frac{(4(k^{3}+3k^{2}+3k+1) + 11(k^{2}+2k+1) + 9(k+1))}{2}$\\\\
$=  \frac{(4(k^{3}+3k^{2}+3k+1) + 11(k^{2}+2k+1) + 9(k+1))}{2}$\\\\
$=  \frac{4k^{3}+12k^{2}+12k+4 + 11k^{2}+22k+11 + 9k+9}{2}$\\\\
$=\frac{4 k^3+ 23 k^2+43 k + 24 }{2}$\\\\




$LHS= RHS \\ QED$


\subsection*{b)}
$ \sum_{m=0}^{n}   2^m = 2^{(n+1)} - 1 $

\subsubsection{Base Case}

test for n=0 (the base case in this question)\\
$ \sum_{m=0}^{0}   2^m = 2^{(0+1)} - 1 $\\
\\
sub in the n=0\\\\
$2^{0}= 2^{1} -1 $\\
$1=2-1$\\
$1 = 1$\\

here we can see the with $n=0$\\ $LHS = RHS$

\subsubsection{Induction Hypothesis}
assume the statement is true when  $n = k $\\
$ \sum_{m=0}^{k}   2^m = 2^{k+1)} - 1 $\\

\subsubsection{Induction }

Proving that the hypothesis holds when $n=k+1$\\



$ LHS =( sum_{m=0}^{k}   2^m)+ 2^{(k+1)} $\\\\
form hypothesis we can substitute in $ \sum_{m=0}^{k}   2^m = 2^{(k+1)} - 1 $\\


from our induction hypothesis we asserted that:\\
$ \sum_{m=0}^{k}   2^m = 2^{(k+1)} - 1 $\\\\
so we substitute it in\\\\
$=\sum_{m=0}^{k+1} =(2^{(k+1)} - 1)+2^{k+1}$ \\\\

we expand and simplify the LHS\\
$= (2^{(k+1)} - 1)+2^{k+1}$\\
$= 2^{k+1}+2^{k+1}-1$\\
$=2 \cdot 2^{k+1} -1$\\
$=2^{k+2} -1$\\\\
do the right hand side\\\\
$RHS =$\\\\
$= 2^{((k+1)+1)}-1$\\\\
	$=2^{(k+2)}-1 $\\\\
$LHS = RHS$\\\\
QED





 
 
 

\end{document}
